\documentclass[11pt]{article}

\usepackage[utf8]{inputenc}
\usepackage[spanish]{babel}
\usepackage[vmargin=2.5cm,hmargin=2.5cm]{geometry}
\usepackage{enumerate}
\usepackage{algpseudocode}

\title{Práctica	1: Robótica \\
Entrega	1: Robot deambulador con ROS y Gazebo}

\author{ Elena María Gómez Ríos }


\begin{document}

\maketitle

\section{Análisis del problema}
La tarea principal de esta práctica es crear un nodo ROS que dirigirá el robot por el entorno simulado de Gazebo siguiendo un algoritmo básico de navegación aleatoria o deambulación, similar al que sigue una aspiradora robot. El robot se moverá hacia adelante hasta que se encuentre cerca de un obstáculo, entonces rotará para evitar el obstáculo y volverá a moverse hacia adelante. La dirección de la rotación estará determinada por la zona en la que el robot vea que hay menos obstáculos.\\

Para calcular la dirección de rotación del robot cuando encuentra un obstáculo obtengo las distancias mínimas del sensor, despreciando la zona central, y calculo la máxima entre la zona de la izquierda y la derecha. El robot rotará, por tanto, hacia la zona con el mayor de estos valores.

\section{Descripción de la solución planteada}
Para implementar el comportamiento descrito arriba me he basado en el nodo \texttt{Stopper} proporcionado por los profesores para la sesión de prácticas.\\

La base principal del programa es la siguiente:

\begin{itemize}
\item \textbf{startMoving()}

Este método se ejecuta en bucle controlando el movimiento del robot. Si no hay un objeto delante el robot sigue avanzando y si hay un objeto delante evalúa cual es la mayor distancia entre las distancias mínimas calculadas para determinar si gira hacia la derecha o izquierda.\\

\begin{algorithmic}
\While {ros::ok()}
	\If {turnTime $> 0$}
    	\State turnTime \-- \--
	\Else
		\If{no hay obstáculos}		
        	\State moveForward() //Avanzo hacia adelante
        	\State puedeGirar = true
        \Else
        	\If{mínimo izquierda $>$ mínimo derecha \textbf{and} puedeGirar}
        		\State puedeGirar = false
        		\State turnTime = 10
        		\State giro = false //Indica que gira hacia la izquierda
        	\ElsIf{puedeGirar}
        		\State puedeGirar = false
        		\State turnTime = 10
        		\State giro = true //Indica que gira hacia la derecha
        	\EndIf
        	\State turn(giro)
		\EndIf
	\EndIf
	\State ros::spinOnce()
    \State rate.sleep()
\EndWhile
\end{algorithmic}

\item \textbf{scanCallback()}

En este método se obtienen los valores del láser con los que analizamos la situación del robot. Si encuentra un obstáculo a una distancia menor que la distancia indicada en el \texttt{launch} el robot se parará y se guarda en una variable el estado. También se obtienen los valores mínimos de las distancias del láser los cuales se utilizan para determinar el giro.

\item \textbf{moveForward()}

Con este método se le indica al robot que se mueva hacia delante y que no gire.

\item \textbf{turn()}

Este método recibe un parámetro que determina el sentido del giro, \texttt{true} quiere decir que gira hacia la derecha y \texttt{false} hacia la izquierda. Mientras realiza el giro el robot no se mueve.



\end{itemize}


\end{document}
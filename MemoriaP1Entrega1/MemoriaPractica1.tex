\documentclass[11pt]{article}

\usepackage[utf8]{inputenc}
\usepackage[spanish]{babel}
\usepackage[vmargin=2.5cm,hmargin=2.5cm]{geometry}
\usepackage{enumerate}
\usepackage{algpseudocode}

\title{Práctica	1: Robótica \\
Entrega	1: Robot deambulador con ROS y Gazebo}

\author{ Elena María Gómez Ríos }


\begin{document}

\maketitle

\section{Análisis del problema}
La tarea principal de esta práctica es crear un nodo ROS que dirigirá el robot por el entorno simulado de Gazebo siguiendo un algoritmo básico de navegación aleatoria o deambulación, similar al que sigue una aspiradora robot. El robot se moverá hacia adelante hasta que se encuentre cerca de un obstáculo, entonces rotará por un tiempo aleatorio y volverá a moverse hacia adelante. La dirección de la rotación estará determinada por la zona en la que el robot vea que hay menos obstáculos.

\section{Descripción de la solución planteada}
Para implementar el comportamiento descrito arriba me he basado en el nodo \texttt{Stopper} de la sesión de prácticas.\\

La base principal del programa es la siguiente:


\begin{algorithmic}
\If {$i\geq maxval$}
    \State $i\gets 0$
\ElsIf {$i+k\leq maxval$}
        \State $i\gets i+k$
\EndIf
\end{algorithmic}


\end{document}